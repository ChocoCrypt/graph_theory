% article example for classicthesis.sty
\documentclass[10pt,a4paper]{article} % KOMA-Script article scrartcl
\usepackage{import}
\usepackage{xifthen}
\usepackage{pdfpages}
\usepackage{transparent}
\newcommand{\incfig}[1]{%
    \def\svgwidth{\columnwidth}
    \import{./figures/}{#1.pdf_tex}
}
\usepackage{lipsum}     %lorem ipsum text
\usepackage{titlesec}   %Section settings
\usepackage{titling}    %Title settings
\usepackage[margin=10em]{geometry}  %Adjusting margins
\usepackage{setspace}
\usepackage{listings}
\usepackage{amsmath}    %Display equations options
\usepackage{amssymb}    %More symbols
\usepackage{xcolor}     %Color settings
\usepackage{pagecolor}
\usepackage{mdframed}
\usepackage[spanish]{babel}
\usepackage[utf8]{inputenc}
\usepackage{longtable}
\usepackage{multicol}
\usepackage{graphicx}
\graphicspath{ {./Images/} }
\setlength{\columnsep}{1cm}

% ====| color de la pagina y del fondo |==== %
\pagecolor{white}
\color{black}



\begin{document}
    %========================{TITLE}====================%
    \title{{  Huffman Encoding  }}
    \author{{Rodrigo Castillo junto a Nicolás Otero}}
    \date{\today}

    \maketitle


    %=======================NOTES GOES HERE===================%

    \section{Huffman Encoding para $TEORIADEGRAFOS$}
        \subsection{creación de la tabla}
        la palabra $TEORIADEGRAFOS$ se puede expresar como un diccionario de la
        forma
        \\
        $dic = $$\{[T,1] [E,2] , [R ,2] , [I,1] , [A,2] , [D,1] , [G,1] , [F,1] ,
        [O,1] , [S,1]\}$
        \\
        teniendo el diccionario, podemos ordenarlo, debido a que hay tantas
        letras con el mismo peso, no importa mucho el orden, sin embargo, el
        orden quedaría así:
        \begin{enumerate}
        \item {$[A,2]$}
        \item {$[E,2]$}
        \item {$[R,2]$}

        \item {$[T,1]$}
        \item {$[I,1]$}
        \item {$[D,1]$}
        \item {$[G,1]$}
        \item {$[F,1]$}
        \item {$[O,1]$}
        \item {$[S,1]$}
        \end{enumerate}

        Para codificar el string $TEORIADEGRAFOS$ vamos a suponer que el string
        está originalmente cifrado en
        \\
        $ASCII(TEORIADEGRAFOS) = $ . $45$ $54$ $52$ $4f$ $41$ $49$ $45$ $44$
        $52$ $47$ $46$ $41$ $53$ $4f$.
        \\
        , ahora recordaremos esto para compararlo con el ejemplo codificado
        mediante el algoritmo de huffman.
        \\
        mediante el algoritmo de huffman, tomaremos el root como el elemento 1
        y construiremos el arbol con respecto a este, por lo tanto .
        \\
        \newpage
        \begin{figure}[ht]
            \centering
            \incfig{arbolhuffman}
            \caption{Arbol de Huffman para string $TEORIADEGRAFOS$}
            \label{fig:arbolhuffman}
        \end{figure}
        por lo tanto ahora el string es $110$ $10$ $1100$ $01$ $100$ $00$   $011$ $10$   $010$ $01$ $00$ $1110$  $1100$ $1000$
        note que :
        \\
        ASCCI(TEORIADEGRAFOS) = $45$ $54$ $52$ $4f$ $41$ $49$    $45$ $44$     $52$ $47$ $46$ $41$ $53$ $4f$
        \\
        BIN(ASCCI(TEORIADEGRAFOS)) = $1000101$ $1010100$ $1010010$ $1001111$ $1000001$
        \\
        $1001001$ $1000101$ $1000100$ $1010010$ $1000111$ $1000110$ $1000001$ $1010011$ $1001111$ :
        \\
        y esto puede verse que es mucho menos pesado que
        \\
        $110$ $10$ $1100$ $01$ $100$ $00$   $011$ $10$   $010$ $01$ $00$ $1110$  $1100$ $1000$
        \\
        sin embargo el string resultante de la codificacion huffman puede no tener sentido como binario , sin embargo, es el siguiente:
        \\
         Codificación de Huffman para el string $TEORIADEGRAFOS=$  $11010110$  $00110000$  $01110010$  $01001110$  $11001000$.












    %=======================NOTES ENDS HERE===================%

    % bib stuff
    \nocite{*}
    \addtocontents{toc}{{}}
    \addcontentsline{toc}{section}{\refname}
    \bibliographystyle{plain}
    \bibliography{../Bibliography}
\end{document}
