% article example for classicthesis.sty
\documentclass[10pt,a4paper]{article} % KOMA-Script article scrartcl
\usepackage{import}
\usepackage{xifthen}
\usepackage{pdfpages}
\usepackage{transparent}
\newcommand{\incfig}[1]{%
    \def\svgwidth{\columnwidth}
    \import{./figures/}{#1.pdf_tex}
}
\usepackage{lipsum}     %lorem ipsum text
\usepackage{titlesec}   %Section settings
\usepackage{titling}    %Title settings
\usepackage[margin=10em]{geometry}  %Adjusting margins
\usepackage{setspace}
\usepackage{listings}
\usepackage{amsmath}    %Display equations options
\usepackage{amssymb}    %More symbols
\usepackage{xcolor}     %Color settings
\usepackage{pagecolor}
\usepackage{mdframed}
\usepackage[spanish]{babel}
\usepackage[utf8]{inputenc}
\usepackage{longtable}
\usepackage{multicol}
\usepackage{graphicx}
\graphicspath{ {./Images/} }
\setlength{\columnsep}{1cm}



\begin{document}
    %========================{TITLE}====================%
\title{{  Taller 3 Teoría de grafos}}
    \author{{Rodrigo Castillo}}
    \date{\today}

    \maketitle


    %=======================NOTES GOES HERE===================%
    \section{Utilice el algoritmo de Ford Fulkerson para encontrar el flujo maximo de la siguiente red...}

\begin{figure}[ht]
    \centering
    \incfig{flujomax}
    \caption{flujomax}
    \label{fig:flujomax}
\end{figure}

\begin{figure}[ht]
    \centering
    \incfig{flujomaxdone}
    \caption{flujomaxdone}
    \label{fig:flujomaxdone}
\end{figure}

    \section{Punto2 }
    % ====|ACA EMPIEZA EL PUNTO 2|====



    % ====|ACA TERMINA EL PUNTO 2|==== %


    \newpage
    \section{podrian 5 casas conectarse con 2 servicios sin cruzar las conexiones?}
    si se puede pues el grafo es plano, además no hay subgrafos de $K_5$ o $K_{3,3}$
    \begin{figure}[ht]
        \centering
        \incfig{conex}
        \caption{conecciones en el grafo plano}
        \label{fig:conex}
    \end{figure}


    \section{Suponga que un grafo simple, conexo , plano $3-regular$ tiene 20
    vertices , ¿cuantas caras tiene el embebimiento libre de cruces del grafo?}
        \textbf{Teorema: C-A+V = 2: }
        como el grafo es 3 regular entonces cada vértice tiene 3 aristas, luego
        $V=20$ , $A=(20*3/2)$ , luego
        \\
        $ C = A -V +2$
        \\
        $C = 30 -20 +2 $
        \\
        $C = 12$.

    \section{Dibujar la triangulación del grafo dado}
        \begin{figure}[ht]
            \centering
            \incfig{triangulacion}
            \caption{triangulacion}
            \label{fig:triangulacion}
        \end{figure}

    \newpage
    \section{Demuestre que el siguiente grafo es plano:}
    tambien se puede decir que es plano puesto que no tiene subgrafos de $K_5$ o $K_{3,3}$
    \begin{figure}[ht]
        \centering
        \incfig{plano}
        \caption{plano}
        \label{fig:plano}
    \end{figure}

    \section{Demuestre que el grafo no es plano}
    \textbf{teorema:un grafo es plano si y solo si no contiene a $K_5$ o $K_{3,3}$.}
    \\
    Note que el grafo dado es $K_6$ , luego contiene a $K_5$ , luego no es
    plano por el teorema anterior.















    %=======================NOTES ENDS HERE===================%

    % bib stuff
    \nocite{*}
    \addtocontents{toc}{{}}
    \addcontentsline{toc}{section}{\refname}
    \bibliographystyle{plain}
    \bibliography{../Bibliography}
\end{document}
