% article example for classicthesis.sty
\documentclass[10pt,a4paper]{article} % KOMA-Script article scrartcl
\usepackage{import}
\usepackage{xifthen}
\usepackage{pdfpages}
\usepackage{transparent}
\newcommand{\incfig}[1]{%
    \def\svgwidth{\columnwidth}
    \import{./figures/}{#1.pdf_tex}
}
\usepackage{lipsum}     %lorem ipsum text
\usepackage{titlesec}   %Section settings
\usepackage{titling}    %Title settings
\usepackage[margin=10em]{geometry}  %Adjusting margins
\usepackage{setspace}
\usepackage{listings}
\usepackage{amsmath}    %Display equations options
\usepackage{amssymb}    %More symbols
\usepackage{xcolor}     %Color settings
\usepackage{pagecolor}
\usepackage{mdframed}
\usepackage[spanish]{babel}
\usepackage[utf8]{inputenc}
\usepackage{longtable}
\usepackage{multicol}
\usepackage{graphicx}
\graphicspath{ {./Images/} }
\setlength{\columnsep}{1cm}

% ====| color de la pagina y del fondo |==== %
\pagecolor{white}
\color{black}



\begin{document}
    %========================{TITLE}====================%
    \title{\rmfamily\normalfont\spacedallcaps{ Parcial 1 Teoría de grafos }}
    \author{\spacedlowsmallcaps{Rodrigo Castillo}}
    \date{\today}

    \maketitle


    %=======================NOTES GOES HERE===================%
    \section{Punto1}
        \subsection{el grafo G:}
            grafo G:
            \begin{figure}[h]
                \centering
                \incfig{grafouno}
                \caption{grafouno}
                \label{fig:G}
            \end{figure}
        \subsection{El grafo H:}
            grafo h:
            \begin{figure}[h]
                \centering
                \incfig{h}
                \caption{H}
                \label{fig:h}
            \end{figure}
        \subsection{Es G complemento $ \equiv   $  H complemento?}
            Por teorema visto en clase, sabemos que G es isomorfo a H si y solo
            si G complemento es isomorfo a H complemento, sin embargo, en H el
            grado máximo de un vértice es 4 mientras que en G es 3 , y esto es
            un invariante, por lo tanto, H no es isomorfo  G, y esto implica
            que H complemento tampoco es isomorfo a G.

            \section{Sea $ G  $  un grafo simple y $ A(G)  $ su matriz de
            adyacencia. Encuentre una manera de escribir $ A(Gcomplemento)  $
            en funcion de $ A(G)  $ . }
            Sea $ G  $  un grafo simple, por lo tanto la matriz de adyacencia $
            A(G)  $ es una matriz de $ 0s  $ y $ 1s  $ .
            \\ por lo tanto, la matriz resultante de invertir los $ 0s  $ y los
            $ 1s  $ de A será la matriz que $ A(Gcomplemento)  $ .
            sea $ A  $ la matriz de adyacencia de un grafo G, para cada $ a_{ij} \in A $ se tiene que:
            \\ si $ a_{ij} = 1 \implies a_{ij} = 0 $
            \\ si $ a_{ij} = 0 \implies a_{ij} = 1  $
            \\ como resultante de esto obtendremos la matriz de $ A(Gcomplemento)  $

            \section{Complete la siguiente tabla}
                \subsection{$ K_4  $ }
                    \subsubsection{numero cromático}
                        como todos los vértices de $ K_4 $ están conectados,
                        entonces el número cromático de $ K_4  $ es 4.
                    \subsubsection{Cintura}
                        si no estoy mal, la cintura es la longitud del ciclo
                        mas corto de un grafo, que en este caso, sería 3
                    \subsubsection{Euleriano}
                        tiene un ciclo euleriano.
                \subsection{$ K_{2,4}  $ }
                    \subsubsection{numero cromático}
                        3
                    \subsubsection{Cintura}
                        4
                    \subsection{Euleriano}
                        tiene un ciclo euleriano
                \subsection{$ P_{2020}  $ }
                    \subsubsection{numero cromático}
                        2020
                    \subsubsection{Cintura}
                        $ ???  $
                    \subsubsection{Euleriano}
                        Circuito



            \section{Si $ u  $ y $ v  $ son los únicos vértices de grado impar
            en un grafo simple $ G  $ entonces $ G  $ contiene un $ u-v camino
            $ .}
                \subsection{Contraejemplo:}
                    Sea $ G  $ el grafo dibujado a continuación , note que $ G
                    $ es un grafo simple en el cuál $ u  $ y $ v  $ son los
                    únicos vértices de grado impar, sin embargo, no contiene
                    una $ u-v  $ caminata y por lo tanto $ G  $ no contiene un $
                    u-v  $ camino.
                    \begin{figure}[ht]
                        \centering
                        \incfig{contraejemplo}
                        \caption{contraejemplo}
                        \label{fig:contraejemplo}
                    \end{figure}









    %=======================NOTES ENDS HERE===================%

    % bib stuff
    \nocite{*}
    \addtocontents{toc}{\protect\vspace{\beforebibskip}}
    \addcontentsline{toc}{section}{\refname}
    \bibliographystyle{plain}
    \bibliography{../Bibliography}
\end{document}
